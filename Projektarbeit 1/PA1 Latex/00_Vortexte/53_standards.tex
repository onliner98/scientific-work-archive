\section{Standards der Lokalisierung}
Globalisierung, Internationalisierung und Lokalisierung umfassen einen weiten Bereich von Aufgaben. Um die Ausführung bestimmter Aufgaben 
\footnote{Beispielsweise den Austausch von Daten zwischen Lokalisierungstools}
zu ermöglichen oder effizienter zu gestalten, wurden seit Gründung der \ac{LISA} verschiedene Standards definiert. Mit Gründung verschiedener Organisationen wurden Standards mit teilweise ähnlichen Zweck definiert. Die definierten Standards sind in diesem Abschnitt aufgeführt, um einen Überblick über diese, ihren Zweck und ihre Relevanz für diese Arbeit zu bieten.
\subsection{TMX}
\ac{TMX} ist ein Standard, der von der \textit{\ac{LISA}} entwickelt wurde und seit deren Insolvenz von der \textit{\ac{GALA}} verwaltet wird. Zweck von \ac{TMX} ist das Bereitstellen einer standardisierten Methode zum Beschreiben von \ac{TM} Daten, welche zwischen Tools und/oder Übersetzungsdienstleistern ausgetauscht werden. Während des Austauschprozesses dürfen keine wichtigen Daten verloren gehen. Dieser Standard wurde im Vorfeld dieser Arbeit recherchiert und stellte sich als nicht relevant für diese Arbeit heraus.
\autocite[Vgl.][]{Savourel.2005}
\subsection{XML:TM}
\ac{XML:TM} ist ein Standard, der von der \textit{\ac{LISA}} entwickelt wurde und seit deren Insolvenz von der \textit{\ac{GALA}} verwaltet wird. Zweck von \ac{XML:TM} ist das Bereitstellen einer standardisierten Methode zum Speicher von \ac{TM} Daten innerhalb eines \ac{XML}-Dokuments. Dieser Standard wurde im Vorfeld dieser Arbeit recherchiert und stellte sich als nicht relevant für diese Arbeit heraus.
\autocite[Vgl.][]{Zydron.2007b}
\subsection{SRX}
\ac{SRX} ist ein Standard, der von der \textit{\ac{LISA}} entwickelt wurde und seit deren Insolvenz von der \textit{\ac{GALA}} verwaltet wird. Zweck von \ac{SRX} ist das Bereitstellen einer standardisierten Methode zum Beschreiben von Segmentierungsregeln, welche zwischen Tools und/oder Übersetzungsdienstleistern ausgetauscht werden. Während des Austauschprozesses dürfen keine wichtigen Daten verloren gehen. Dieser Standard wurde im Vorfeld dieser Arbeit recherchiert und stellte sich als nicht relevant für diese Arbeit heraus.
\autocite[Vgl.][]{Pooley.2008}
\subsection{GMX-V}
\ac{GMX-V} ist ein Standard, der von der \textit{\ac{LISA}} entwickelt wurde und seit deren Insolvenz von der \textit{\ac{GALA}} verwaltet wird. Zweck von \ac{GMX-V} ist das Definieren einer Metrik zum eindeutigen Messen einer gegebenen globalen Informationsverwaltungsaufgabe. Des Weiteren beschreibt \ac{GMX-V} eine standardisierte Methode zum Austauschen der Metriken zwischen Tools und/oder Übersetzungsdienstleistern. Während des Austauschprozesses dürfen keine wichtigen Daten verloren gehen. Dieser Standard wurde im Vorfeld dieser Arbeit recherchiert und stellte sich als nicht relevant für diese Arbeit heraus.
\autocite[Vgl.][]{Zydron.2007}
\subsection{TBX}
\ac{TBX} ist ein Standard, der von der \textit{\ac{LISA}} entwickelt wurde und seit deren Insolvenz von der \textit{\ac{GALA}} verwaltet wird. Zweck von \ac{TBX} ist das Bereitstellen einer standardisierten Methode zum Beschreiben terminologischer Daten, welche zwischen Tools und/oder Übersetzungsdienstleistern ausgetauscht werden. Während des Austauschprozesses dürfen keine wichtigen Daten verloren gehen. Dieser Standard wurde im Vorfeld dieser Arbeit recherchiert und stellte sich als nicht relevant für diese Arbeit heraus.
\autocite[Vgl.][S. vii]{GALA.2008}
\subsection{XLIFF}
\ac{XLIFF} ist ein Standard, der von der \textit{\ac{OASIS}} entwickelt wurde und seither von dieser verwaltet wird. Zweck von \ac{XLIFF} ist das Bereitstellen einer standardisierten Methode zum Speichern von Lokalisierungsdaten und deren Austausch zwischen einzelnen Prozessschritten der Lokalisierung. Während des Austauschprozesses dürfen keine wichtigen Daten verloren gehen. Ziel des Standards ist es, Kompatibilität zwischen verschiedenen Tools zu ermöglichen. Dieser Standard wurde im Vorfeld dieser Arbeit recherchiert und stellte sich als relevant für diese Arbeit heraus.
\autocite[Vgl.][]{Schnabel.2014}
\subsection{Trans-WS}
\ac{Trans-WS} ist ein Standard, der von der \textit{\ac{OASIS}} entwickelt wurde und seither von dieser verwaltet wird. Zweck von \ac{Trans-WS} ist das Bereitstellen einer standardisierten Methode zur Benutzung und Beschreibung eines Webservices innerhalb der Übersetzungsindustrie. Dieser Standard wurde im Vorfeld dieser Arbeit recherchiert und stellte sich als nicht relevant für diese Arbeit heraus.
\autocite[Vgl.][]{Reynolds.2006}
\subsection{UTX}
\ac{UTX} ist ein Standard, der von der  \textit{\ac{AAMT}} entwickelt wurde und seither von dieser verwaltet wird. Zweck von \ac{UTX} ist das Bereitstellen eines standardisierten lexikalischen Formats für regelbasierte Übersetzungssoftware und das Bereitstellen eines Formats zur Darstellung von Glossaren, welche in den Bereichen von \ac{CAT} und \ac{NLP} Eingesetzt werden. Dieser Standard wurde im Vorfeld dieser Arbeit recherchiert und stellte sich als nicht relevant für diese Arbeit heraus.
\autocite[Vgl.][S. 4]{AAMT.2018}
\subsection{OLIF}
\ac{OLIF} ist ein Standard, der von dem \textit{\ac{OLIF} Consortium} entwickelt wurde und seither von diesem verwaltet wird. Zweck von \ac{OLIF} ist das Bereitstellen einer standardisierten Methode zum Beschreiben von lexikalischen und terminologischen Daten, welche zwischen Tools und/oder Übersetzungsdienstleistern ausgetauscht werden. Während des Austauschprozesses dürfen keine wichtigen Daten verloren gehen. Dieser Standard wurde im Vorfeld dieser Arbeit recherchiert und stellte sich als nicht relevant für diese Arbeit heraus.
\autocite[Vgl.][]{McCormick.2002}
\subsection{ITS}
\ac{ITS} ist ein Standard, der von dem \textit{\ac{W3C}} entwickelt wurde und seither von diesem verwaltet wird. Zweck von \ac{ITS} ist das Bereitstellen einer \ac{XML}-basierten Auszeichnung von Metadaten im Bereich der Internationalisierung, Übersetzung und Lokalisierung. Dieser Standard wurde im Vorfeld dieser Arbeit recherchiert und stellte sich als relevant für diese Arbeit heraus.
\autocite[Vgl.][]{Filip.2013}