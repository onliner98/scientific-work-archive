\chapter*{Kurzfassung}
%\begingroup
%\begin{table}[h!]
%\setlength\tabcolsep{0pt}
%\begin{tabular}{p{3.7cm}p{11.7cm}}
%Titel & \DerTitelDerArbeit \\
%Verfasser/in: & \DerAutorDerArbeit \\
%Kurs: & \DieKursbezeichnung \\
%Ausbildungsstätte: & \DerNameDerFirma\\
%\end{tabular}
%\end{table}
%\endgroup
%Einführung und Motivation
Softwareprodukte werden heutzutage weltweit vermarktet. Infolgedessen müssen Unternehmen ihre Softwareprodukte effizient lokalisieren, falls sie sich erfolgreich auf dem Weltmarkt behaupten wollen. 
\autocite[Vgl.][S. 1]{Schmitz.2000}
\autocite[Vgl.][S. 1]{Reineke.2005} 
Aus diesem Grund ist Software zur Automatisierung der Lokalisierung nötig. 
%Forschungslücke
Derzeit wird von Movilizer keine Software zur Lokalisierung von Movelets eingesetzt. 
%Ziel
Ziel dieser Arbeit ist die Konzeption einer Software, welche zu übersetzende und übersetzte Strings der Texte eines Movelets automatisiert verarbeitet. Diese Software wird im Folgenden als Lokalisierungstool bezeichnet. 
%Methode
Anhand von Literaturrecherche werden verschiedene Konzepte analysiert und diskutiert. 
%Ergebnisse
Das resultierende Lokalisierungstool extrahiert zu übersetzende Strings in XLIFF Dateien anhand von ITS und einer $\_()$ Auszeichnung. Die extrahierten Strings erhalten eine Übersetzung, welche von dem Lokalisierungstool genutzt wird um die ausgezeichneten Strings zu ersetzen. 
%Interpretation
Resultierend daraus ist dieses Lokalisierungstool der Anfang der Automatisierung der Lokalisierung von Movelets.