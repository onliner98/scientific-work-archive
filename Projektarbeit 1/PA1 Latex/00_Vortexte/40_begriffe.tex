\chapter{Begriffe}
\section{Globalisierung}
Als Globalisierung bezeichnet man alle Aktivitäten eines Unternehmens mit dem Ziel der Vermarktung eines (Software-)Produkts außerhalb des nationalen, lokalen Marktes. Zu diesem Zweck sind technische, wirtschaftliche und gesetzliche Aspekte des Zielmarktes besonders zu berücksichtigen. Infolgedessen ist die Globalisierung im Kontext der betriebswirtschaftlichen und kaufmännischen Unternehmensführung zu betrachten.
\autocite[Vgl.][S. 1]{Reineke.2005}
\autocite[Vgl.][ S. 2]{Schmitz.2000}
\section{Internationalisierung}
\label{sec:internationalisierung}
Als Internationalisierung bezeichnet man die Entwicklung von (Software-)Produkten auf eine Weise, die eine möglichst schnelle Lokalisierung mit geringen Aufwand ermöglicht. Zu diesem Zweck muss das (Software-)Produkt mit einer Funktionalität entwickelt werden, die eine Anpassung an technische Konventionen, kulturelle Eigenheiten und Sprache des Zielmarktes ermöglichen. Infolgedessen ist die Internationalisierung immer im Kontext der Entwicklung zu betrachten.
\autocite[Vgl.][S. 2]{Reineke.2005}
\autocite[Vgl.][ S. 2]{Schmitz.2000}
\section{Lokalisierung}
\label{sec:lokalisierung}
Als Lokalisierung bezeichnet man die eigentliche Anpassung von (Software-)Produkten an technische Konventionen, kulturelle Eigenheiten und Sprache des Zielmarktes. Zu diesem Zweck müssen (Software-)Produkte zunächst internationalisiert werden, um die Lokalisierung zu ermöglichen.
\autocite[Vgl.][S. 2]{Reineke.2005}
\autocite[Vgl.][ S. 3]{Schmitz.2000}
In der Konsequenz ist der Aufwand der Lokalisierung geringer, je mehr Aufwand in die Internationalisierung investiert wird.
\autocite[Vgl.][S. 2]{Reineke.2005}
\section{Abgrenzung von Übersetzung, Lokalisierung und Internationalisierung}
Ist ein Text in einer Ausgangssprache verfasst, so wird das Verfassen dieses Texts in einer Zielsprache als Übersetzen bezeichnet. In Abschnitt \ref{sec:lokalisierung} ist die Anpassung von Sprache an einen Zielmarkt als Teil der Lokalisierung definiert. Des Weiteren ist in Abschnitt \ref{sec:internationalisierung} das Entwickeln von Software mit Funktionalität zur einfachen Anpassung an Sprache als Teil der Internationalisierung definiert. Die Anpassung von Sprache beinhaltet unter anderem das Übersetzen von Texten. Daher ist das Übersetzen ein Teil der Lokalisierung. Daraus leitet sich wiederum ab, dass das Übersetzen ein Teil der Lokalisierung und der Internationalisierung ist.
\section{Translation Memory} 
\ac{TM} ist eine Datenbank voneinander zugehörigen Textpaaren. Die Texte des Textpaares sind in unterschiedlichen Sprachen vorhanden, der Ausgangssprache und der Zielsprache. Das Textpaar besteht aus einem Text in der Ausgangssprache und der Übersetzung dieses Texts in der Zielsprache. Folglich sind die Übersetzungen zwecks Wiederverwendung gespeichert. Diese Wiederverwendung ermöglicht eine Steigerung der Übersetzungsgeschwindigkeit und der Kosteneffizienz, besonders bei repetitiven Texten.
\autocite[Vgl.][S. 187]{OBrien.2010}
\section{Computer Assisted Translation}
Der Einsatz von Software zum Zweck der automatischen Übersetzung erzielt keine hochqualitativen Übersetzungsergebnisse. Eine Möglichkeit, die Qualität automatisch übersetzter Texte zu erhöhen, ist die nachträgliche Verbesserung durch Übersetzer. Bei diesem Prozess kann die eingesetzte Software jedoch nicht von dem Wissen der Übersetzer profitieren. Aus diesem Grund ist es sinnvoll, Übersetzer in der Zusammenarbeit mit Software während des Übersetzungsprozesses einzusetzen. Dies wird als \ac{CAT} bezeichnet. Auf diese Weise werden lexikalische, syntaktische und semantische Mehrdeutigkeiten in der Interaktion von Mensch und Software gelöst. Resultierend daraus steigen Übersetzungsqualität und Übersetzungsgeschwindigkeit.
\autocite[Vgl.][S. 4]{Barrachina.2009}
\section{Natural Laguage Processing}
\ac{NLP} ist ein interdisziplinäres Feld, welches Algorithmen und Systeme zum Verstehen und Verarbeiten natürlicher Sprache durch Computer entwickelt und erforscht. Aufgabe von \ac{NLP} ist es menschliche Sprache in gesprochener und geschriebener Form zu analysieren und infolgedessen Kommandos und nützliche Informationen aus dieser zu extrahieren.
\autocite[Vgl.][S. 135]{Sintoris.2017}
\section{Auszeichnung}
Textverarbeitungssysteme benötigen typischerweise zusätzliche Information innerhalb des zu verarbeitenden Dokuments. Diese zusätzliche Information wird in natürlichen Text eingebettet und als Auszeichnung 
\footnote{Im Englischen Markup}
bezeichnet. Diese Auszeichnung dient dem Zweck der Trennung der logischen Elemente des Dokuments von den Regeln zur Verarbeitung der jeweiligen Elemente.
\autocite[Vgl.][S. 1]{Ellison.1994}
\section{Extensible Markup Language}
\ac{XML} ist eine Auszeichnungssprache, welche Daten als Inhalte von Elementen mit Tags und Attributen auszeichnet. Diese Elemente sind hierarchisch in einer Baumstruktur geordnet und für Menschen lesbar. Der Aufwand der Implementierung einer \ac{XML}"~erstellenden und/oder -verarbeitenden Software soll gering sein. Des Weiteren soll \ac{XML} direkt über das Internet genutzt werden können. Aufgrund dessen ermöglicht \ac{XML} das plattform"~unabhängige, einfache Speichern und Austauschen von Daten für Software.
\autocite[Vgl.][]{Bray.2008}
\section{Movilizer}
Movilizer ist der Server für den operativen Arbeitsbereich. Movilizer bietet die zentralisierte Vernetzung aller mobilen Geräte und Geschäftsprozessen eines Unternehmens. Zusätzlich ist Movilizer der Name des Unternehmens, welches Movilizer entwickelt und vertreibt.
\autocite[Vgl.][]{Nitschkowski.2018c}
\section{Movelet}
Das Movelet ist eine zentrale Komponente von Movilizer, welches einen Geschäftsprozess innerhalb eines Geschäftsszenarios abbildet. Das Movelet beinhaltet alle Informationen, welche zur Ausführung dieses Geschäftsprozesses auf einem mobilen Gerät nötig sind. Movelets können kombiniert werden um komplexe und umfassende Geschäftsprozesse auf einer einfach zu nutzenden mobilen Anwendung abzubilden. Von einem technischen Standpunkt sind Movelets eine universelle Basiseinheit, welche auf einer XML-Struktur basieren.
\autocite[Vgl.][]{Nitschkowski.2018c}
\section{Stammdaten}
Stammdaten sind Datenpakete, welche zum Datentransfer zwischen Datenbanken, Movilizer Servern und Movelets verwendet werden. Stammdaten können von Movelets verarbeitet und angezeigt werden. Zu diesem Zweck enthalten Stammdaten Texte, Nummern oder Binärdateien, wie beispielsweise Bilder. Diese Inhalte sind in einer hierarchischen \ac{XML}-Struktur organisiert.
\autocite[Vgl.][]{Nfitschkowski.2018d}
\section{Movilizer Extensible Markup Language}
\ac{MXML} ist eine \ac{XML}-basierte Auszeichnungssprache, welche alle XML-Element und -Attribute im Kontext von Movilizer beschreibt.
\autocite[Vgl.][]{Nitschkowski.2015}
\section{Movilizer Expression Language}
\ac{MEL} ist eine auf Ereignissen basierende Programmiersprache, welche in die \ac{MXML}-Struktur eines Movelets integriert ist. In der Konsequenz ermöglicht \ac{MEL} das Nutzen von Ereignissen zur Verarbeitung von Benutzereingaben, manipulieren der Benutzeroberfläche und weiteren dynamischen Elementen innerhalb von Movelets. 
\autocite[Vgl.][]{Nitschkowski.2018e}
\section{Movilizer Gradle Plug"~in} 
Das Movilizer Gradle Plug"~in ist ein Gradle Plug"~in mit der Aufgabe Movelet Projekte von der Entwicklung bis zur Veröffentlichung zu unterstützen. Das Movilizer Gradle Plug"~in bietet zwei Prozesse, den \mbox{\textit{Compile Request}} und den \mbox{\textit{Send Request}}. Der \mbox{\textit{Compile Request}} ruft alle in einem Movelet referenzierten Dateien ab und erstellt anhand dieser eine Anfrage für die Movilizer Server. Der \mbox{\textit{Send Request}} sendet die von dem \mbox{\textit{Compile Request}} erstellten Anfragen mit den benötigten Einstellungen an die Movilizer Server. Des Weiteren ermöglicht das Movilizer Gradle Plug"~in das Nutzen einer Software zur Verarbeitung von Vorlagen. Diese Software ermöglicht unter anderem das Verwenden von Variablensubstitutionen, Kommentaren, konditionalen Blöcken, Schleifen und Importen von externen Quelltexten innerhalb der Vorlage.
\autocite[Vgl.][]{Mula.2018}
\section{Abfragesprache} 
Eine Abfragesprache ist eine Sprache, in welcher eine Abfrage an ein Informationssystem gestellt werden kann. Ziel dieser Abfrage ist es gesuchte Informationen zu erhalten.
\autocite[Vgl.][S7f]{Reiner.1991}
Eine Abfragesprache besteht aus den getrennten Definitionen ihrer Syntax und Semantik. Die Syntax definiert das für die Abfrage zur Verfügung stehende Vokabular und seiner erlaubten Kombinationen. Die Semantik definiert die Interpretationsvorschriften einer Abfrage und damit, welches Ergebnis diese zurückliefert.
\autocite[Vgl.][S. 24]{Willenborg.2001}