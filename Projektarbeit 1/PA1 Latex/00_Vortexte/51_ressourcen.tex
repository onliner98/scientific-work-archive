\section{Trennung von Quelltext und Ressourcen}
Eine der ersten Formen der Internationalisierung von Software, ist die strikte Trennung des Quelltexts von den Ressourcen. Hierbei müssen alle Strings in Ressource Dateien getrennt vom Quelltext gespeichert werden. Die Trennung muss also bereits beim Beginn der Softwareentwicklung erfolgen. Dies ermöglicht die Lokalisierung der Software durch Austauschen der Ressource Dateien mit andere Ressource Dateien, die  eine anderen Sprache enthalten. Die Ressource Dateien bestehen aus Schlüssel-Wert-Paaren. Die Schlüssel enthalten die Bezeichner, welche die Stellen im Quelltext referenzieren, an denen die zugehörigen Werte eingefügt werden. Die Werte enthalten die Strings in der Zielsprache. Um die Ressource Dateien zu übersetzen, müssen also lediglich die Werte übersetzt werden. Die Bezeichner bleiben gleich. Das Übersetzen geschieht in der Regel nicht durch Ändern der originalen Ressource Dateien. Stattdessen werden Kopien erzeugt, welche lediglich den Schlüssel enthalten. In diesen neuen Ressource Dateien werden die Strings in der Zielsprache als Wert eingefügt. Beim Kompilieren des Quelltexts werden die Bezeichner durch die zugehörigen Strings ersetzt. Resultieren daraus wird eine übersetzte Version der Software erzeugen.
\autocite[Vgl.][S. 145]{Reineke.2005}
\par
Diese Form der Internationalisierung ist in dem Kontext von Movelets möglich. Alle String Ressourcen eines in dieser Form internationalisierten Movelets sind in Stammdaten ausgelagert. Die Stammdaten beinhalten die Schlüssel-Wert-Paare, bestehend aus Strings und deren zugehörige Übersetzungen in den benötigten Zielsprachen. Wie bereits erwähnt, muss jedoch diese Trennung von Beginn der Softwareentwicklung berücksichtigt werden, während ein Lokalisierungstool auch nach fortgeschrittener Programmierung des Movelets ohne große Quelltextänderungen eingesetzt werden kann. Dies wird im folgenden Abschnitt erläutert.