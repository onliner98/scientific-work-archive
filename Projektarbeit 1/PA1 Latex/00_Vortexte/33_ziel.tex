\section{Ziel dieser Projektarbeit}
Ziel dieser Arbeit ist die Konzeption einer Software, welche zu übersetzende und übersetzte Strings der Texte eines Movelets automatisiert verarbeitet. Resultierend daraus verringert die Software den Aufwand des Prozesses der Lokalisierung. Aufgrund dessen, dass die Software einen Unterprozess der Lokalisierung automatisiert, wird sie im Folgenden als Lokalisierungstool bezeichnet. 
\par
Weitere Unterprozesse der Lokalisierung werden von diesem Lokalisierungstool nicht umgesetzt. Hierzu zählen:
\begin{itemize}
	\item Übersetzung
	\item Verwendung geeigneter Zeichensätze
	\item Übersetzung verknüpfter Strings
	\item Anpassung von Grafiken und Symbolen
	\item Anpassen der Benutzeroberfläche an Textlängen und Flussrichtungen
	\item Anpassung an geltendes Recht
	\item Anpassung der Formate für Adresse, Datum et cetera
	\item Anpassung der Maßeinheiten für Gewicht, Währung, et cetera
	\item Anpassung der Sortierung und Suche
\end{itemize}
\autocite[Vgl.][S. 423]{HassellCorbiell.2001}
\autocite[Vgl.][]{Asnes.2010}
\autocite[Vgl.][]{Wagner.2017}
\par
Im Verlauf dieser Arbeit werden zunächst die sich aus dem Ziel ergebenden Anforderungen formuliert. Des Weiteren, werden die möglichen Lösungsansätze zu den einzelnen Anforderungen formuliert und unter Berücksichtigung vorausgegangener Erkenntnisse diskutiert. Die Lösungsansätze ergeben sich anhand von Literaturrecherche. Sollten sich aus der Diskussion neue Anforderungen ergeben, wird dieses Verfahren iterativ angewandt. Aus der Gesamtheit aller erarbeiteten Lösungen ergibt sich das Konzept des Lokalisierungstools. 