\chapter{Anhang}

\section{Implementierung der Extraktion zu übersetzender Strings anhand einer Auszeichnung}
\label{sec:implementierung}
Zu dem Zweck der Implementierung eines Lokalisierungstools, welches die ausgezeichneten Strings extrahiert, ist für \ac{ITS} ein Namespace-based Validation Dispatching Language 
\autocite[Vgl.][]{ISO.2006}
Dokument gegeben.
\autocite[Vgl.][]{Filip.2013}
Die Auszeichnung \textit{\_()} lässt sich durch die Produktion einer kontextfreien Grammatik wie in \ref{eq:grammar} dargestellt definieren. Die dargestellte Produktion ist teil einer kontextfreie Grammatik, welche auch eine reguläre Grammatik ist, sie wird in der Augmented Backus"~Nauer Form 
\autocite[Vgl.][]{Crocker.2008} 
dargestellt.
\begin{align} \label{eq:grammar}
	\begin{array}{lcl}
		Q & = & \text{'''''} \text{ } | \text{ } \text{''''''} \\
		WS & = & (\epsilon \text{ } | \text{ } WSP \text{ } | \text{ } CR \text{ } |\text{ } LF)^* \\ 
		STRING & = & (VCHAR \text{ } | \text{ } DIGIT \text{ } |WS)^* \\
		MARKUP0 & = & \text{''\_(''} \text{ } WS \text{ } STRING \text{ } WS \text{ } \text{'')''} \\ 
		& & \text{; 0 steht für alle ohne ' ausgezeichneten Strings} \\
		MARKUP1 & = & \text{''\_(''} \text{ } WS \text{ } Q \text{ } STRING \text{ } Q \text{ } WS \text{ } \text{'')''} \\
		& & \text{; 1 steht für alle mit ' ausgezeichneten Strings} \\
		MARKUP2 & = & \text{''\_(''} \text{ } WS \text{ } \text{''concat(''}  \text{ } WS  \text{ } (CALL \text{ } \\
		& & \text{; 2 steht für alle concatinierten Strings} \\
		& | & \text{ } Q \text{ } STRING \text{ } Q ) \text{ } (WS  \text{ } \text{'',''} \text{ } WS \text{ } (CALL \text{ } \\
		& | & \text{ } Q \text{ } STRING \text{ } Q ) \text{ } )^+ \text{ } WS \text{ } \text{'')''} \text{ } WS  \text{ } \text{'')''} \\
		CALL & = & CHAR \text{ } STRING^* \\
		& & \text{ ; Aufruf von Variable oder Methode}
	\end{array}
\end{align}