\chapter{Diskussion}
Im Verlauf dieser Arbeit wurden die Anforderungen an das Lokalisierungstool anhand des Ziels dieser Arbeit formuliert. Für diese Anforderungen wurden Lösungsansätze anhand der Literatur recherchiert und diskutiert. Infolgedessen lassen sich Anforderungen und Lösungsansätze anhand der angegebenen Literatur nachvollziehen. Die Entscheidung für den jeweiligen Lösungsansatz wurde anhand einer Diskussion belegt.
\par
Ziel dieser Arbeit ist die Konzeption einer Software, welche zu übersetzende und übersetzte Strings der Texte eines Movelets automatisiert verarbeitet. Das erarbeitete Lokalisierungstool automatisiert die Extraktion ausgezeichneter Strings und das Ersetzen dieser durch deren Übersetzungen. Auf diese Weise wird ein hoher Grad der Automatisierung erzielt. Die Auszeichnung und Übersetzung der zu übersetzenden Strings erfolgt jedoch manuell und bietet folglich weitere Möglichkeit der Automatisierung. Resultierend daraus ist dieses Lokalisierungstool der Anfang einer Automatisierung der Lokalisierung von Movelets. Des Weiteren sind zum Zweck der Implementierung und Verwendung des Lokalisierungstools weitere Aspekte zu beachten. Diese sind die Konzeption der Benutzeroberfläche des Lokalisierungstools, die Auswahl der Übersetzer, beziehungsweise der Übersetzungssoftware und eine Dateistruktur für generierten \ac{XLIFF} Dateien. Zur Konzeption der Benutzeroberfläche zählt insbesondere das Auflisten und Behandeln von Kollisionen.
\par
Der Umfang dieser Arbeit ist aufgrund des von Movilizer vorgegebenen Zieles und dem vorgesehenen Umfang einer Projektarbeit begrenzt. Die Lokalisierung selbst jedoch umfasst weit mehr Aspekte. Zu diesen zählen.
\begin{itemize}
	\item Übersetzung
	\item Verwendung geeigneter Zeichensätze
	\item Übersetzung verknüpfter Strings
	\item Anpassung von Grafiken und Symbolen
	\item Anpassen der Benutzeroberfläche an Textlängen und Flussrichtungen
	\item Anpassung an geltendes Recht
	\item Anpassung der Formate für Adresse, Datum et cetera
	\item Anpassung der Maßeinheiten für Gewicht, Währung, et cetera
	\item Anpassung der Sortierung und Suche
\end{itemize}
In der Konsequenz bietet die Lokalisierung viel weiteres Automatisierungspotential für Movilizer. Bezogen auf dieses Lokalisierungstool selbst lassen sich weiter Forschungsaspekte empfehlen. Hierzu zählt die automatisierte Übersetzung der generierten \ac{XLIFF} Dateien. Anstelle von einer automatisierten Übersetzung bietet sich die Konzeption der Möglichkeit einer Übersetzung in der Benutzeroberfläche des Movelets selbst an. Dies ermöglicht die Verfügbarkeit alle Kontextinformationen bezüglich der Übersetzung. Des Weiteren ist es wissenswert, wie der Einsatz der in den \ac{XLIFF} Dateien gespeicherten Übersetzungen zum Aufbau einer Terminologiedatenbank oder eines \ac{TM} möglich ist, welche die Übersetzungskosten senken und die Übersetzungsqualität steigern.