\chapter{Conclusion}
\label{chp:conclusion}
In the context of the emerging market for augmented reality solutions for field workers \autocites{EY.2019a}{EY.2019b}{Detzel.2018}{Shook.2019}{Guy.2019}, this paper investigated a sub-task of perceiving the environment of field workers, i.e., tool image classification. Tool image classification was investigated by determining the best-performing neural network for tool image classification in the course of an experiment. This paper found that, in general, several neural networks are suitable for tool image classification. Especially, ResNet-152, ResNeXt-101, and DenseNet-264 were proven to be suitable for tool image classification. 
%
For industry, the resulting concept of the literature review conducted by this paper and their implementations can be used as a basis for implementing software perceiving the environment of field workers and, thus, as a basis for implementing augmented reality solutions for field workers.
%
Furthermore, this paper introduced a novel tool image classification dataset called \ac{TIC Dataset}. This paper hopes to foster further research in the field of computer vision by providing the \ac{TIC Dataset} publicly available under the Creative Commons Attribution Share Alike 4.0 International License.


%- summarize in regard to intro (Context, Problem, Relevance)
%  - Restate Hypothesis
%  - Summarize Main Results
%  - Einordnung in Context: 
%    - Why do the main results matter? => CONTRIBUTION(to industry/research)
%    - Returning to your problem statement to explain how your research helps solve the problem.
%    - Referring back to the literature review and showing how you have addressed a gap in knowledge.
%    - Discussing how your findings confirm or challenge an existing theory or assumption.
%  - Handlungsempfehlung
%  - Closure sentence: broarder question, warning, or call to action.
%- DONT SUMMARIZE WHAT YOU HAVE DONE!!! DAS IST AUFGABE DES ABSTARCTS