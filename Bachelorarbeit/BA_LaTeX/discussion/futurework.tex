This paper proposes future work in regard to the limitations of this paper. 
% business case
As stated in Section \ref{sec:motivation}, an augmented reality solution for field workers requires software perceiving the environment of field workers. Therefore, this paper proposes to investigate the implementation of a holistic solution including other computer vision tasks, a software framework, and a hardware framework.
\par % semu/unsipervised, learning auxiliaries
As discussed in Section \ref{sec:limitations}, semi-supervised learning, unsupervised learning, and learning auxiliaries, i.e., transfer learning, adversarial training, data augmentation, input normalization, weight decay, and multi-task learning, might improve performance. On that account, this paper proposes to investigate the effect of each of those techniques.
\par % hyperparam/architect search, versions of sota nns, non neural models
As discussed in Section \ref{sec:limitations}, regarding  neural architecture search, hyperparameter search, different versions of the state-of-the-art neural networks for image classification, and non-neural models to conduct the experiment might reveal even better-performing models for tool image classification than found in the results of this paper. Hence, this paper proposes to investigate these models.
\par
As discussed in Section \ref{sec:limitations}, the \ac{TIC Dataset} can be extended. To train machine learning algorithms, training data is required. \autocite{ElAmir.2020} As a result, extending the \ac{TIC Dataset} might help fostering future research in the field of computer vision.
Thus, this paper proposes to extend the \ac{TIC Dataset} by creating more images as described in Section \ref{sec:datasetconstruction}, adding subsets containing tool images of already existing datasets, and adding additional annotations for object detection, image segmentation or other computer vision tasks.
\par
In the experiment conducted by this paper, EfficientNet-B7 did not learn to classify tool images. As discussed, the accuracy of EfficientNet-B7 reported in Chapter \ref{chp:result} does not reflect the performance of EfficientNet-B7 for tool image classification. As stated in Section \ref{sec:disceffnet}, this paper hypothesizes that the small batch size is the cause. Accordingly, this paper proposes to repeat the experiment on hardware capable of supporting a larger batch size for EfficientNet-B7. If this proves to be insufficient, this paper proposes to reduce regularization as well.