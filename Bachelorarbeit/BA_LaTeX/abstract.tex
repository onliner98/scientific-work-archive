\chapter*{Abstract}
\begingroup
\begin{table}[h!]
\setlength\tabcolsep{0pt}
\begin{tabular}{p{3.7cm}p{11.7cm}}
Title & \DerTitelDerArbeit \\
Author: & \DerAutorDerArbeit \\
Course of Study: & \DieKursbezeichnung \\
Company: & \DerNameDerFirma\\
\end{tabular}
\end{table}
\endgroup

% Problemstellung (Einführung und Motivation/Relevanz)
The augmented reality market is estimated to reach 83.5 billion USD by 2021. \autocite{Statista.2019} According to Boston Consulting Group, Accenture, Mc Kinsey, and others, augmented reality solutions for field workers are an important field within that market. \autocites{EY.2019a}{EY.2019b}{Detzel.2018}{Shook.2019}{Guy.2019} An augmented reality solution for field workers requires software perceiving the environment of field workers. A sub-task of that perception is to classify tools of different classes.
% Forschungsziel
This paper determines the best-performing neural network for tool image classification.
% Methode
The best-performing neural network for tool image classification is determined in the course of an experiment. The experiment trains and evaluates state-of-the-art neural networks for image classification on a dataset constructed by this paper. The state-of-the-art neural networks for image classification are determined in the course of a literature review conducted by this paper.
% Ergebniss
This paper found that, in general, not only one neural network is suitable for tool image classification, but several neural networks are suitable for tool image classification. Especially, ResNet-152, ResNeXt-101, and DenseNet-264 were proven to be suitable for tool image classification.
% Resultierende Empfehlung / Interpretation
Furthermore, this paper introduces a novel dataset for tool image classification called \ac{TIC Dataset}. This paper hopes to foster further research in the field of computer vision by providing the \ac{TIC Dataset} publicly available under the Creative Commons Attribution Share Alike 4.0 International License.

%- nur konkretes
%- 1-2 Sätze je Punkt

